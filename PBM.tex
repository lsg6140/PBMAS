\documentclass[12pt]{article}
\usepackage[vmargin=2cm,hmargin=2cm]{geometry}
\usepackage{graphicx}
\usepackage{listings}
\usepackage{courier}
\usepackage{amsmath}
\usepackage{amssymb}
\usepackage{breqn}
\usepackage{bm}
\usepackage{color}

\newcommand{\R}{\mathbb{R}}
\newcommand{\N}{\mathbb{N}}

\begin{document}
\section{Population balance model}
\subsection{Breakage}
The model for breakage is given as
\begin{equation*}
\frac{\partial n}{\partial t}=\int_l^\infty S(\epsilon)b(l,\epsilon)n(\epsilon)d\epsilon-S(l)n(l)
\end{equation*}
where $S$ is a selection rate constant and $b$ is a breakage function. Number density function $n(l)$ gives the number of particles with $l\in(l,l+dl)$ as $dN=n(l)dl\\$.

\subsubsection{Discretized breakage birth}
\begin{equation}
R_i^{[1]}=\sum_{j=i}^{n}b_{i,j}S_jN_j
\end{equation}
where $S_i$ is the selection rate for interval $i$ and $b_{i,j}$ is the number of fragments from $j$ to $i$ which occurs in $1\sim n$
\subsubsection{Discretized breakage death}
\begin{equation}
R_i^{[2]}=S_iN_i
\end{equation}
which occurs in $2\sim n$
\subsubsection{Discretized selection rate}
The average number of fragments produced by breaking granule of size $l$ is
\begin{equation}
N_b(l)=\int_0^lb(x,l)dx
\end{equation}
The overall rate of generation of numbers is
\begin{equation}
\begin{aligned}
R_0&=\int_0^\infty\overline{B}_0^B(l)-\overline{D}_0^B(l)dl\\
&=\int_0^\infty\left[N_b(l)-1\right]S(l)n(l)dl
\end{aligned}
\end{equation}
Discrete eqivalent is
\begin{equation}
\begin{aligned}
R_0&=\sum_{i=1}^n(B_i^B-D_i^B)\\
&=\sum_{i=2}^n-S_iN_i+\sum_{i=1}^n\sum_{j=i}^nb_{i,j}S_jN_j\\
&=\sum_{i=2}^n-S_iN_i+\sum_{j=1}^n\sum_{i=1}^jb_{i,j}S_jN_j\\
&=\sum_{i=2}^n-S_iN_i+\sum_{i=1}^n\sum_{j=1}^ib_{j,i}S_iN_i\\
&=\sum_{i=1}^nS_iN_i\left(\sum_{j=1}^ib_{j,i}-1\right)+S_1N_1
\end{aligned}
\end{equation}
For the continuous and discrete equations to be equivalent,
\begin{equation}
\begin{aligned}
\int_{l_i}^{l_{i+1}}[N_b(l)-1]S(l)n(l)dl=S_iN_i\left(\sum_{j=1}^ib_{j,i}-1\right)\\
\int_{l_1}^{l_2}[N_b(l)-1]S(l)n(l)dl=b_{1,1}S_1N_1
\end{aligned}
\end{equation}
Assume the simple relationship,
\begin{equation}
n(l)=\frac{N_i}{l_{i+1}-l_i}
\end{equation}
then
\begin{equation}
S_i=\frac{\frac{1}{l_{i+1}-l_i}\int_{l_i}^{l_{i+1}}\left[N_b(l)-1\right]S(l)dl}{\sum_{j=1}^ib_{j,i}-1}
\end{equation}
Since it is assumed that there is no more breakage in the interval $(0,l_1)$,
\begin{equation}
S_1=0.
\end{equation}

\subsubsection{Discretized breakage function}
Consider the movement of particle volume from one interval to another. The rate of generation of volume of fragments from interval $i$ is
\begin{equation}
\int_{l_i}^{l_{i+1}}l^3S(l)n(l)dl
\end{equation}
with discretized form of
\begin{equation}
\overline{l}_i^3N_iS_i
\end{equation}
The number of particles of size $x$ produced by the breakage of particle of size $l$ is
\begin{equation*}
n(x) = S(l)n(l)b(x,l)
\end{equation*}
The volume of particles of size $x$ is
\begin{equation*}
v(x)=x^3S(l)n(l)b(x,l)
\end{equation*}
The volume of particles of size in $j$th term is
\begin{equation*}
v_j=\int_{l_j}^{l_{j+1}}x^3S(l)n(l)b(x,l)dx
\end{equation*}
Therefore, fragments arrive in the interval $j$ from interval $i$ at a rate
\begin{equation}
\begin{aligned}
R_{j,i}=&\int_{l_i}^{l_{i+1}}\int_{l_j}^{l_{j+1}}x^3S(l)n(l)b(x,l)dxdl,\qquad j<i\\
R_{i,i}=&\int_{l_i}^{l_{i+1}}\int_{l_i}^lx^3S(l)n(l)b(x,l)dxdl
\end{aligned}
\end{equation}
with discretized form of
\begin{equation}
\overline{l}_j^3b_{j,i}N_iS_i
\end{equation}
Therefore, volume will be apportioned appropriately to the intervals if
\begin{equation*}
\left(\frac{\overline{l}_j}{\overline{l}_i}\right)^3b_{j,i}=\frac{\int_{l_i}^{l_{i+1}}\int_{l_j}^{l_{j+1}}x^3S(l)n(l)b(x,l)dxdl}{\int_{l_i}^{l_{i+1}}l^3S(l)n(l)dl}
\end{equation*}


\begin{equation}
b_{j,i}\approx\left(\frac{\overline{l}_i}{\overline{l}_j}\right)^3\frac{\int_{l_i}^{l_{i+1}}\int_{l_j}^{l_{j+1}}x^3S(l)b(x,l)dxdl}{\int_{l_i}^{l_{i+1}}l^3S(l)dl}
\end{equation}

\begin{equation}
b_{i,i}\approx\frac{\int_{l_i}^{l_{i+1}}\int_{l_i}^lx^3S(l)b(x,l)dxdl}{\int_{l_i}^{l_{i+1}}l^3S(l)dl}
\end{equation}

\subsection{Log-normal distribution of breakage function}
The probability distribution of volume of particle size of $x$ produced by the breakage of particle size of $l$ is $P(x|l)$. Then
\begin{equation} \label{eq:prob}
\int_0^lP(x|l)dx=1
\end{equation}
This means sum of all particles' volume is same with the volume of original particle. Assuming that volume of original particle $l^3$ and that of broken particle $x^3$ the breakage function which is the number of particles produced by the breakage of original particle is
\begin{equation}
b(x,l)=\left(\frac{l}{x}\right)^3P(x|l)
\end{equation}
For mass or volume conservation, sum of all particle volumes generated by a particle of size $l$ should be $l^3$. That is
\begin{equation}
\int_0^lx^3b(x,l)dx=l^3
\end{equation}
which is just equivalent with eq. \ref{eq:prob}.\\
Deconvolution of particle size distribution (PSD) of activted sludge shows clear modes of log-normal distribution so that $\ln(L)\sim N(\mu,\sigma)$. Hence, the breakage function is
\begin{equation}
b(x,l)=\left(\frac{l}{x}\right)^3\frac{\frac{1}{x\sigma\sqrt{2\pi}}\exp\left(-\frac{\left(\ln x-\mu\right)^2}{2\sigma^2}\right)}{\frac{1}{2}\text{erfc}\left(-\frac{\ln l-\mu}{\sqrt{2}\sigma}\right)}
\end{equation}

\subsection{Selection function}
Selection rate function by size dependent
\begin{equation}
S(l)=S_0l^p
\end{equation}
If $p=0$, the the selection function is size independent, $p=1,~2,~3$ indicates that the selection function is dependent on the length, area, and volume of particles, respectively.

Mimimum values of sum of squared errors (SSE) for each $p$ are given below 

\begin{center}
	\begin{tabular}{ |c|c|c|c|c| } 
		\hline
		p & 0 & 1 & 2 & 3 \\
		\hline 
		SSE & 0.047 & 0.021 & 0.015 & 0.014 \\ 
		\hline
	\end{tabular}
\end{center}
The parameters for p=3 are 2.68517472e-07, 6.49856613e-01, 2.24167522e-01.
Hence, it can be concluded that the selection function is dependent on the volume of particles.

\subsection{Selection function with critical length}
\begin{equation}
S(x) = \frac{S_0}{1+\exp\left[-k(x-x_c)\right]}
\end{equation}
Estimated parameters are S0 = 1.54542348e-01, R1 = 6.88138159e-01, R2 = 2.04767439e-01, k = 6.57566833e-02,
Xc = 7.20512415e+01.
The SSE is 0.0065.

\end{document}